% ch10.tex
% Dieses Werk ist unter einem Creative Commons Namensnennung-Keine kommerzielle Nutzung-Weitergabe
% unter gleichen Bedingungen 3.0 Deutschland Lizenzvertrag lizenziert. Um die Lizenz anzusehen, gehen Sie bitte
% zu http://creativecommons.org/licenses/by-nc-sa/3.0/de/ oder schicken Sie einen Brief an
% Creative Commons, 171 Second Street, Suite 300, San Francisco, California 94105, USA.


%\chapter{Where to go from here}
\chapter{Wie geht's jetzt weiter}

%Congratulations! You've made it to the end.
Gratulation! Du hast es bis hierher geschafft.
\par
%What you've hopefully learned from this book, are basic concepts that will make learning other programming languages much simpler.  While Python is a brilliant programming language, one language is not \emph{always} the best tool for every task.  So don't be afraid of looking at other ways to program your computer, if it interests you.
\vspace{1em}
Hoffentlich hast du ein paar grundlegende Konzepte gelernt, die dir auch das Lernen von anderen Programmiersprachen leichter machen werden. Während Python eine tolle Programmiersprache ist, ist Python nicht \emph{immer} das beste Werkzeug für jede Aufgabe. Schau dir immer wieder mal andere Sprachen und Möglichkeiten an, deinen Computer zu programmieren.

%For example, if you're interested in games programming, you can perhaps look at something like BlitzBasic (\href{http://www.blitzbasic.com}{www.blitzbasic.com}), which uses the Basic programming language. Or perhaps Flash (which is used by many websites for animation and games---for example, the Nickelodeon website, \href{http://www.nick.com}{www.nick.com}, uses a lot of Flash).
\vspace{1em}
Wenn dich Spieleprogrammierung ganz besonders interessiert, dann könntest du vielleicht einen Blick auf BlitzBasic (\href{http://www.blitzbasic.com}{www.blitzbasic.com}) werfen. Oder vielleicht auf Flash, mit dem viele Internetseiten Animationen und Spiele machen---wie zum Beispiel die Nickelodeon-Seite \href{http://www.nick.com}{www.nick.com}.

%If you're interested in programming Flash games, possibly a good place to start would be `Beginning Flash Games Programming for Dummies', a book written by Andy Harris, or a more advanced reference such as `The Flash 8 Game Developing Handbook' by Serge Melnikov.  Searching for `flash games' on \href{http://www.amazon.com}{www.amazon.com} will find a number of books on this subject.

%Some other games programming books are: `Beginner's Guide to DarkBASIC Game Programming' by Jonathon S Harbour (also using the Basic programming language), and `Game Programming for Teens' by Maneesh Sethi (using BlitzBasic). Be aware that BlitzBasic, DarkBasic and Flash (at least the development tools) all cost money (unlike Python), so Mum or Dad will have to get involved before you can even get started.

%If you want to stick to Python for games programming, a couple of places to look are: \href{http://www.pygame.org}{www.pygame.org}, and the book `Game Programming With Python' by Sean Riley.
\vspace{1em}
Falls du weiterhin Python für die Spieleentwicklung verwenden willst, ist die Seite \href{http://www.pygame.org}{www.pygame.org} ein guter Tipp.

%If you're not specifically interested in games programming, but do want to learn more about Python (more advanced programming topics), then take a look at `Dive into Python' by Mark Pilgrim (\href{http://www.diveintopython.org}{www.diveintopython.org}).  There's also a free tutorial for Python available at: \href{http://docs.python.org/tut/tut.html}{http://docs.python.org/tut/tut.html}.  There's a whole pile of topics we haven't covered in this basic introduction so, at least from the Python perspective, there's still a lot for you to learn and play with.
\vspace{1em}
Und wenn du einfach mehr über Python erfahren und lernen willst (die fortgeschrittenen Programmierthemen), könntest du das frei verfügbare Python Tutorial auf (\href{http://tutorial.pocoo.org/}{http://tutorial.pocoo.org/} durchlesen oder dich auf der deutschsprachigen Python Seite \href{http://wiki.python.de}{http://wiki.python.de} umsehen.

%\par\par\noindent
%\emph{Good luck and enjoy your programming efforts.}
\vspace{1em}
\emph{Viel Glück, und habe Spaß beim Programmieren.}

\newpage
